%% Palatino font:
\usepackage[T1]{fontenc}
\usepackage[utf8]{inputenc}
\usepackage{palatino}


\usepackage{pifont}
\DeclareUnicodeCharacter{2730}{\ding{80}}
\DeclareUnicodeCharacter{2605}{\ding{72}}
\DeclareUnicodeCharacter{2055}{\ding{105}}  

% \DeclareUnicodeCharacter{274D}{\ding{114}}
% \DeclareUnicodeCharacter{2730}{\ding{80}}
\DeclareUnicodeCharacter{274D}{\ding{96}}

% for correct quotes in verbatim
\usepackage{upquote}

\pagestyle{plain} % no running header

% \usepackage{tcolorbox}
%\newtcolorbox{myquote}{colback=black!10!white, colframe=black!40!white}
% \newtcolorbox{myquote}{colback=Goldenrod!20!white, colframe=Goldenrod!60!white, arc=1.0mm, boxrule=0.5mm}


\usepackage[many]{tcolorbox}
\definecolor{quotegray}{HTML}{505050}
\newtcolorbox{myquote}{%
    enhanced jigsaw, 
    breakable,      % allow page breaks
    frame hidden,   % hide the default frame
    left=1cm,       % left margin
    right=1cm,      % right margin
    colback=white,
    fontupper=\color{quotegray},
    overlay={%
        \node [scale=4,
            text=lightgray,
            inner sep=0pt,] at ([xshift=0.5cm,yshift=-0.7cm]frame.north west){``}; 
        \node [scale=4,
            text=lightgray,
            inner sep=0pt,] at ([xshift=-0.5cm]frame.south east){''};  
            },
        % paragraph skips obeyed within tcolorbox
                parbox=false,
}

% redefine the 'quote' environment to use this 'myquote' environment
\renewenvironment{quote}{\begin{myquote}}{\end{myquote}}







\usepackage{framed} % not sure i need this anymore

\usepackage[T1]{fontenc}
\usepackage{inconsolata}
% \usepackage{ascii}

% \definecolor{myshade}{rgb}{1.0 1.0 1.0}
% % \definecolor{myshade}{rgb}{0.655, 0.498, 0.161}
% \newtcolorbox{mybox}{colback=myshade!100!white, colframe=myshade!100!white, arc=0.3mm, boxrule=0.1mm, left=1mm}

% % \newtcolorbox{mybox}{colback=myshade!100!white, colframe=myshade!100!white, arc=0.3mm, boxrule=0.1mm, left=1mm}
% \newtcolorbox{mybox}{colback=\textcolor[rgb]{0.655, 0.498, 0.161}, colframe=\textcolor[rgb]{0.655, 0.498, 0.161}, arc=0.3mm, boxrule=0.1mm, left=1mm}

% I have to only define Shaded if it is already defined.
% The reason is that pandoc does not define the macro if there are not code blocks in the a markdown file.
\ifx \@Shaded \@empty

% \renewenvironment{Shaded}{\begin{mybox}}{\end{mybox}}

% \definecolor{codecolor}{RGB}{0, 0, 0}

% % background color for code blocks
% \let\OldTexttt\texttt
% \renewcommand{\texttt}[1]{\OldTexttt{\colorbox{myshade}{\textcolor[rgb]{0, 0, 0}{{\small #1}}}}}

\renewcommand{\KeywordTok}[1]{\textcolor[rgb]{0, 0, 0}{\textbf{{#1}}}} % def and or not reg
\renewcommand{\BuiltInTok}[1]{\textcolor[rgb]{0.373, 0.298, 0.580}{\textbf{{#1}}}} % print open 
%\renewcommand{\VariableTok}[1]{\textcolor[rgb]{0.500, 0.500, 0.500}{\textbf{{#1}}}} % print open 
\renewcommand{\VariableTok}[1]{\textcolor[rgb]{0.141, 0.392, 0.824}{\textbf{{#1}}}}
\renewcommand{\OperatorTok}[1]{\textcolor[rgb]{0, 0, 0}{{#1}}} % def and or not reg
\renewcommand{\DataTypeTok}[1]{\textcolor[rgb]{1.0,0.13,0.00}{{#1}}}
\renewcommand{\DecValTok}[1]{\textcolor[rgb]{0.655, 0.498, 0.161}{{#1}}}
\renewcommand{\BaseNTok}[1]{\textcolor[rgb]{0.259, 0.592, 0.596}{{#1}}}
\renewcommand{\FloatTok}[1]{\textcolor[rgb]{0.655, 0.498, 0.161}{{#1}}}
%\renewcommand{\CharTok}[1]{\textcolor[rgb]{0.259, 0.592, 0.596}{{#1}}}
\renewcommand{\CharTok}[1]{\textcolor[rgb]{0.678,0.141,0.098}{{#1}}}
\renewcommand{\StringTok}[1]{\textcolor[rgb]{0.678,0.141,0.098}{{#1}}}
\renewcommand{\CommentTok}[1]{\textcolor[rgb]{0.135, 0.134, 0.133}{{#1}}}
%\renewcommand{\CommentTok}[1]{\textcolor[rgb]{0.796,0.188,0.282}{{#1}}} % made comments same color as strings, so they are not colored differently when used as comments.
\renewcommand{\OtherTok}[1]{\textcolor[rgb]{0.00,0.44,0.13}{{#1}}}
\renewcommand{\AlertTok}[1]{\textcolor[rgb]{1.00,0.00,0.00}{\textbf{{#1}}}}
\renewcommand{\FunctionTok}[1]{\textcolor[rgb]{0.549, 0.102, 0.063}{\textbf{{#1}}}}  % function name
\renewcommand{\RegionMarkerTok}[1]{{#1}}
\renewcommand{\ErrorTok}[1]{\textcolor[rgb]{1.00,0.00,0.00}{\textbf{{#1}}}}
%\renewcommand{\NormalTok}[1]{\textcolor[rgb]{0.361,0.451,0.502}{\textbf{{#1}}}}
%\renewcommand{\NormalTok}[1]{\textcolor[rgb]{0.361,0.451,0.502}{{#1}}}

% \renewcommand{\NormalTok}[1]{\textcolor[rgb]{0, 0, 0}{{\fontsize{8}{9} \selectfont #1}}}
\renewcommand{\NormalTok}[1]{\textcolor[rgb]{0, 0, 0}{{#1}}}


\else
  % no code blocks with markup...
\fi


%  % KMT only use subsubsection number (this increments though the book 
%  % when we supress numbering with {-} after each header except Exerisices
\renewcommand\thesubsubsection{\arabic{section}}
\renewcommand\thesubsubsection{\arabic{subsection}}
\renewcommand\thesubsubsection{\arabic{chapter}-\arabic{subsubsection}}

% put number after subsubsection title
\makeatletter
\renewcommand\sectionlinesformat[4]{%
  \Ifstr{#1}{subsubsection}
    {\@hangfrom{\hskip #2}{#4 #3}}
    {\@hangfrom{\hskip #2#3}{#4}}% original definition for other section levels
}
\makeatother
\renewcommand\sectionformat{\enskip\thesection\autodot}


\renewcommand*{\chapterformat}{%
  \textcolor[rgb]{0.8, 0.8, 0.8}{\thechapter}\autodot\enskip%
}
% \renewcommand*{\sectionformat}{%
%   \textcolor[rgb]{0.655, 0.498, 0.161}{\thesection}\autodot\enskip%
% }
% \renewcommand*{\subsectionformat}{%
%   \textcolor[rgb]{0.655, 0.498, 0.161}{\thesubsection}\autodot\enskip%
% }
\renewcommand*{\subsubsectionformat}{%
  \textcolor[rgb]{0.8, 0.8, 0.8}{\thesubsubsection}\autodot\enskip%
}


% % prevent latex from "floating" the figures
% \usepackage{float}
% \let\origfigure\figure
% \let\endorigfigure\endfigure
% \renewenvironment{figure}[1][2] {
%     \expandafter\origfigure\expandafter[H]
% } {
%     \endorigfigure
% }
